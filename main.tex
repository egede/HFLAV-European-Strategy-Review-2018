\pdfoutput=1
\documentclass[12pt,a4paper]{article}

% Variables that controls behaviour
\usepackage{ifthen} % for conditional statements
\newboolean{pdflatex}
\setboolean{pdflatex}{true} % False for eps figures 

\newboolean{articletitles}
\setboolean{articletitles}{true} % False removes titles in references

\newboolean{uprightparticles}
\setboolean{uprightparticles}{false} %True for upright particle symbols

\newboolean{inbibliography}
\setboolean{inbibliography}{false} %True once you enter the bibliography

% Define titles and authors here. It will then be used both in metadata and in
% what is printed on the front page.
\def\paperauthors{The Heavy Flavor Averaging Group (HFLAV)} % Leave as is for PAPER and CONF
\def\paperasciititle{HFLAV submission to The European Strategy for Particle Physics} % Set ASCII title here
\def\papertitle{HFLAV submission to The European Strategy for Particle Physics} % Latex formatted title
\def\paperkeywords{{High Energy Physics}, {HFLAV}, {Flavour physics}} % Comma separated list
\def\papercopyright{2018 HFLAV} % new since 9/Apr/2018
\def\paperlicence{CC-BY-4.0 licence}
\def\paperlicenceurl{https://creativecommons.org/licenses/by/4.0/}

\input{preamble}
\usepackage{bm}

\begin{document}
%%%%%%%%%%%%%%%%%%%%%%%%%
%%%%% Title     %%%%%%%%%
%%%%%%%%%%%%%%%%%%%%%%%%%

\setcounter{page}{1}
\thispagestyle{empty}
\renewcommand\Affilfont{\itshape\small}

\title{
  \vspace*{-1cm}
  \papertitle
\vskip0.20in
\large{\it \paperauthors}
%%\author{set one author to remove the spurious ``immediate''}
\vspace*{-0.20in}}

% Authors -------------------------------------------------
\author[1]{Y.~Amhis}\affil[1]{LAL, Universit\'{e} Paris-Sud, CNRS/IN2P3, Orsay, France}
\author[2]{Sw.~Banerjee}\affil[2]{University of Louisville, Louisville, Kentucky, USA}
\author[3]{E.~Ben-Haim}\affil[3]{LPNHE, Universit\'{e} Pierre et Marie Curie, Universit\'{e} Paris Diderot, CNRS/IN2P3, Paris, France}
\author[39]{M.~Bona}\affil[39]{TBD}
\author[5]{A.~Bozek}\affil[5]{H. Niewodniczanski Institute of Nuclear Physics, Krak\'{o}w, Poland}
\author[6]{C.~Bozzi}\affil[6]{Universita e INFN, Ferrara, Ferrara, Italy}
\author[39]{J.~Brodzicka}
\author[5,7]{M.~Chrz\k{a}szcz}\affil[7]{Physik-Institut, Universit\"at Z\"urich, Z\"urich, Switzerland}
\author[4]{J.~Dingfelder}
\author[23]{U.~Egede}
\author[8]{M.~Gersabeck}\affil[8]{School of Physics and Astronomy, University of Manchester, Manchester, UK}
\author[9]{T.~Gershon}\affil[9]{Department of Physics, University of Warwick, Coventry, UK}
\author[39]{L.~Gibbons}
\author[11]{P.~Goldenzweig}\affil[11]{Institut f\"{u}r Experimentelle Kernphysik, Karlsruher Institut f\"{u}r Technologie, Karlsruhe, Germany}
\author[13]{K.~Hayasaka}\affil[13]{Niigata University, Niigata, Japan}
\author[14]{H.~Hayashii}\affil[14]{Nara Women's University, Nara, Japan}
\author[39]{D.~Johnson}
\author[15]{M.~Kenzie}\affil[15]{Cavendish Laboratory, University of Cambridge, Cambridge, UK}
\author[16]{T.~Kuhr}\affil[16]{Ludwig-Maximilians-University, Munich, Germany}
\author[10]{O.~Leroy}
\author[39]{H.~Li}
\author[17,18]{A.~Lusiani}\affil[17]{Scuola Normale Superiore, Pisa, Italy}\affil[18]{Sezione INFN di Pisa, Pisa, Italy}
\author[39]{V.~L\"{u}th}
\author[13]{K.~Miyabayashi}
\author[20]{P.~Naik}\affil[20]{H.H.~Wills Physics Laboratory, University of Bristol, Bristol, UK}
\author[21]{T.~Nanut}\affil[21]{J. Stefan Institute, Ljubljana, Slovenia}
\author[22]{A.~Oyanguren Campos}\affil[22]{Instituto de Fisica Corpuscular, Centro Mixto Universidad de Valencia - CSIC, Valencia, Spain}
\author[23]{M.~Patel}\affil[23]{Imperial College London, London, UK}
\author[39]{A.~Pompili}
\author[18]{M.~Rama}
\author[26]{M.~Roney}\affil[26]{University of Victoria, Victoria, British Columbia, Canada}
\author[27]{M.~Rotondo}\affil[27]{Laboratori Nazionali dell'INFN di Frascati, Frascati, Italy}
\author[28]{O.~Schneider}\affil[28]{Institute of Physics, Ecole Polytechnique F\'{e}d\'{e}rale de Lausanne (EPFL), Lausanne, Switzerland}
\author[29]{C.~Schwanda}\affil[29]{Institute of High Energy Physics, Vienna, Austria}
\author[30]{A.~J.~Schwartz}\affil[30]{University of Cincinnati, Cincinnati, Ohio, USA}
\author[31,32]{B.~Shwartz}\affil[31]{Budker Institute of Nuclear Physics (SB RAS), Novosibirsk, Russia}\affil[32]{Novosibirsk State University, Novosibirsk, Russia}
\author[10]{J.~Serrano}
\author[39]{A.~Soffer}
\author[18]{D.~Tonelli}
\author[36]{P.~Urquijo}\affil[36]{School of Physics, University of Melbourne, Melbourne, Victoria, Australia}
\author[37]{R.~Van Kooten}\affil[37]{Indiana University, Bloomington, Indiana, USA}
\author[38]{J.~Yelton}\affil[38]{University of Florida, Gainesville, Florida, USA}

\maketitle

{\footnotesize 
% Edit macro in main.tex to keep metadata correct
\centerline{\copyright~\papercopyright. \href{\paperlicenceurl}{\paperlicence}.}}
\vspace*{1cm}

\renewcommand{\thefootnote}{\arabic{footnote}}
\setcounter{footnote}{0}


%%%%%%%%%%%%%%%%%%%%%%%%%
%%%%% Main text %%%%%%%%%
%%%%%%%%%%%%%%%%%%%%%%%%%

\pagestyle{plain} % restore page numbers for the main text
\setcounter{page}{1}
\pagenumbering{arabic}

%% Uncomment during review phase. 
%% Comment before a final submission.
\linenumbers

\noindent The Heavy Flavor Averaging Group (HFLAV)~\cite{HFLAV16} was formed in 2002 to 
continue the activities of the LEP Heavy Flavor Steering 
Group~\cite{Abbaneo:2000ej_mod,*Abbaneo:2001bv_mod_cont}. 
The group is responsible for calculating averages of measurements of beauty and charm hadron properties from all experiments across the globe. With this mandate, the members of HFLAV are well 
placed to comment on the future importance of flavour physics. \textbf{The submission here is making the argument for the support of heavy flavour physics in both the near term and for the far future. Specific recommendations are in bold.}

For the next decade the majority of new measurements will come
from the \lhcb experiment, following upgrade I, and from the \belletwo experiment. The \lhcb experiment
will benefit from extremely large sample sizes of all types of $b$-hadrons while \belletwo will have an
environment where the full kinematics of a \B-meson decay is controlled from the known initial state. 
To give examples of their complementarity, \lhcb will provide the most accurate measurements of angular
distributions in electroweak penguin decays with leptons, while the \belletwo experiment will measure the branching fraction
of the decay $\decay{\Bz}{\Kstarz \neu\neub}$. The combination of these two analyses is placing very strong
limits on the nature of physics involving new couplings to leptons, whereas they individually are not able to do so. \textbf{The dual approach of conducting heavy flavour physics in both the controlled} $\bm{\ep \en}$ \textbf{environment and in the much higher statistics hadronic environment should be supported into the future.}

Following on the from first precision $b$ physics measurements in a hadron collider environment at the
\tevatron, the \lhcb experiment has pushed the boundaries of what can be done at a hadron collider well 
beyond its original goals. The physics case for setting limits on, or indeed exploring physics beyond the SM
in heavy flavour decays is stronger than ever. \lhcb upgrade II~\cite{LHCb-PII-Physics} is 
proposing to create an experiment that will be able to collect data at a luminosity a factor two above the luminosity of upgrade I. It will lead to large jumps in the
sensitivity in key  areas such as \CP violation in the charm and beauty systems, angular observables in
penguin decays and the study of lepton universality. The mass scale for new physics that can be explored will grow by close to a factor two compared to the situation prior to the \lhc high luminosity operation. The theoretical uncertainties related to the measurements are in all key channels under control. \textbf{Support for the construction of 
\lhcb upgrade II and its exploitation though the complete high luminosity \lhc period is essential.}

Measurements in the last few years from heavy flavour physics has given rise to questions about if decays of
hadrons to final states with leptons are purely governed by SM physics. In particular the measurements of the
angular distribution in the $\decay{\Bd}{\Kstarz\mup\mun}$ decay, and the comparison of the branching fractions
of the decays $\decay{B}{K^{(\ast)}\mup\mun}$ and $\decay{B}{K^{(\ast)}\ep\en}$ ($R_{K}$, $R_\Kstar$) have led
to a large number of models that introduce either new gauge bosons or leptoquarks. The explosion in the
searches for direct production of these new particles within \atlas and \cms in the last few years have
illustrated how heavy flavour physics informs and motivates the search for new physics through the direct
production of new states. It has also been shown that a new hadron collider with a 100\tev collision 
energy will be able to discover new particles in the majority of possible masses and couplings in models that
can explain the recent results on lepton non-universality~\cite{Allanach:2017bta}. \textbf{It is paramount that any new facility under consideration should give thoughts to how heavy flavour 
physics can
be further explored there.} For a future circular collider this could be a flavour physics experiment along the
lines of \lhcb, a fixed target facility or the capability of the experiments to fully explore the flavour
physics potential of collecting a very large sample of Z boson decays. The abundance of measurements that have
been possible to be carried out at \lhcb makes it clear that \textbf{there is a huge advantage of having a
dedicated
flavour physics experiment with the detector and trigger optimised for this, compared to having a flavour
physics programme running as a parasitic activity of a high} $\bm \pt$ \textbf{experiment.}

The strategy of particle physics is not only about which new facilities should be funded but also about funding
the research to be carried out at those facilities. Across Europe there are 61 groups involved in heavy flavour
physics on the \lhcb experiment, YY on the \belletwo experiment, 17 on the \besiii experiment, and a
significant number on \atlas and \cms. In addition to this there is a large number
of theory groups working on the phenomenology of heavy flavour decays. The data analyses in these
experiments are 
often performed in relatively small groups at universities and government labs and can take multiple 
years to
finish due to their high complexity. The analyses also typically involve detailed and specific collaboration
with phenomenologists in terms of obtaining information that is optimised for interpretation. 
For these reasons
the long term support of the experimental and theoretical groups involved in flavour physics is an essential
part of exploiting the investment in the facilities and associated experiments. \textbf{The European strategy should make it clear
that supporting particle physics is not only a matter of supporting CERN and other global laboratories 
but also about supporting the individual experimental and theoretical research groups.}

\setboolean{inbibliography}{true}
\bibliographystyle{LHCb}
\bibliography{standard}
 
\end{document}
